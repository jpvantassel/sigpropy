%% Generated by Sphinx.
\def\sphinxdocclass{report}
\documentclass[letterpaper,10pt,english,openany,oneside]{sphinxmanual}
\ifdefined\pdfpxdimen
   \let\sphinxpxdimen\pdfpxdimen\else\newdimen\sphinxpxdimen
\fi \sphinxpxdimen=.75bp\relax

\PassOptionsToPackage{warn}{textcomp}
\usepackage[utf8]{inputenc}
\ifdefined\DeclareUnicodeCharacter
% support both utf8 and utf8x syntaxes
  \ifdefined\DeclareUnicodeCharacterAsOptional
    \def\sphinxDUC#1{\DeclareUnicodeCharacter{"#1}}
  \else
    \let\sphinxDUC\DeclareUnicodeCharacter
  \fi
  \sphinxDUC{00A0}{\nobreakspace}
  \sphinxDUC{2500}{\sphinxunichar{2500}}
  \sphinxDUC{2502}{\sphinxunichar{2502}}
  \sphinxDUC{2514}{\sphinxunichar{2514}}
  \sphinxDUC{251C}{\sphinxunichar{251C}}
  \sphinxDUC{2572}{\textbackslash}
\fi
\usepackage{cmap}
\usepackage[T1]{fontenc}
\usepackage{amsmath,amssymb,amstext}
\usepackage{babel}



\usepackage{times}
\expandafter\ifx\csname T@LGR\endcsname\relax
\else
% LGR was declared as font encoding
  \substitutefont{LGR}{\rmdefault}{cmr}
  \substitutefont{LGR}{\sfdefault}{cmss}
  \substitutefont{LGR}{\ttdefault}{cmtt}
\fi
\expandafter\ifx\csname T@X2\endcsname\relax
  \expandafter\ifx\csname T@T2A\endcsname\relax
  \else
  % T2A was declared as font encoding
    \substitutefont{T2A}{\rmdefault}{cmr}
    \substitutefont{T2A}{\sfdefault}{cmss}
    \substitutefont{T2A}{\ttdefault}{cmtt}
  \fi
\else
% X2 was declared as font encoding
  \substitutefont{X2}{\rmdefault}{cmr}
  \substitutefont{X2}{\sfdefault}{cmss}
  \substitutefont{X2}{\ttdefault}{cmtt}
\fi


\usepackage[Bjarne]{fncychap}
\usepackage{sphinx}

\fvset{fontsize=\small}
\usepackage{geometry}

% Include hyperref last.
\usepackage{hyperref}
% Fix anchor placement for figures with captions.
\usepackage{hypcap}% it must be loaded after hyperref.
% Set up styles of URL: it should be placed after hyperref.
\urlstyle{same}
\addto\captionsenglish{\renewcommand{\contentsname}{Contents:}}

\usepackage{sphinxmessages}
\setcounter{tocdepth}{1}



\title{SigProPy}
\date{Nov 12, 2019}
\release{0.0.1}
\author{Joseph P.\@{} Vantassel}
\newcommand{\sphinxlogo}{\vbox{}}
\renewcommand{\releasename}{Release}
\makeindex
\begin{document}

\pagestyle{empty}
\sphinxmaketitle
\pagestyle{plain}
\sphinxtableofcontents
\pagestyle{normal}
\phantomsection\label{\detokenize{index::doc}}



\chapter{Summary}
\label{\detokenize{index:summary}}
SigProPy is Python module for digital signal processing. The module includes
two main class definitons \sphinxtitleref{TimeSeries} and \sphinxtitleref{FourierTransform}. These classes
include various methods for creating and manipulating time series and Fourier
transforms.


\chapter{License Information}
\label{\detokenize{index:license-information}}\begin{quote}

Copyright (C) 2019 Joseph P. Vantassel (\sphinxhref{mailto:jvantassel@utexas.edu}{jvantassel@utexas.edu})

This program is free software: you can redistribute it and/or modify
it under the terms of the GNU General Public License as published by
the Free Software Foundation, either version 3 of the License, or
(at your option) any later version.

This program is distributed in the hope that it will be useful,
but WITHOUT ANY WARRANTY; without even the implied warranty of
MERCHANTABILITY or FITNESS FOR A PARTICULAR PURPOSE.  See the
GNU General Public License for more details.

You should have received a copy of the GNU General Public License
along with this program.  If not, see \textless{}https: //www.gnu.org/licenses/\textgreater{}.
\end{quote}


\chapter{TimeSeries Class}
\label{\detokenize{index:timeseries-class}}\index{TimeSeries (class in sigpropy)@\spxentry{TimeSeries}\spxextra{class in sigpropy}}

\begin{fulllineitems}
\phantomsection\label{\detokenize{index:sigpropy.TimeSeries}}\pysiglinewithargsret{\sphinxbfcode{\sphinxupquote{class }}\sphinxbfcode{\sphinxupquote{TimeSeries}}}{\emph{amplitude}, \emph{dt}, \emph{n\_stacks=1}, \emph{delay=0}}{}
A class for manipulating time series.
\begin{description}
\item[{Attributes:}] \leavevmode\begin{description}
\item[{amp}] \leavevmode{[}ndarray{]}
Denotes the time series amplitude. If \sphinxtitleref{amp} is 1D each 
sample corresponds to a single time step. If \sphinxtitleref{amp} is 2D 
each row corresponds to a particular section of the time
record (i.e., time window) and each column corresponds to a
single time step.

\item[{dt}] \leavevmode{[}float {]}
Denotes the time step between samples in seconds.

\item[{n\_windows}] \leavevmode{[}int{]}
Number of time windows that the time series has been split
into (i.e., number of rows of amp if 2D).

\item[{n\_samples}] \leavevmode{[}int{]}
Number of samples in time series (i.e., \sphinxtitleref{len(amp)} if \sphinxtitleref{amp}
is 1D or number of columns if \sphinxtitleref{amp} is 2D).

\item[{fs}] \leavevmode{[}float{]}
Sampling frequency in Hz equal to \sphinxtitleref{1/dt}.

\item[{fnyq}] \leavevmode{[}float{]}
Nyquist frequency in Hz equal to \sphinxtitleref{fs/2}.

\end{description}

\end{description}
\index{\_\_init\_\_() (TimeSeries method)@\spxentry{\_\_init\_\_()}\spxextra{TimeSeries method}}

\begin{fulllineitems}
\phantomsection\label{\detokenize{index:sigpropy.TimeSeries.__init__}}\pysiglinewithargsret{\sphinxbfcode{\sphinxupquote{\_\_init\_\_}}}{\emph{amplitude}, \emph{dt}, \emph{n\_stacks=1}, \emph{delay=0}}{}
Initialize a TimeSeries object.
\begin{description}
\item[{Args:}] \leavevmode\begin{description}
\item[{amplitude}] \leavevmode{[}ndarray {]}
Amplitude of the time series at each time step. Refer to
attribute definition for details.

\item[{dt}] \leavevmode{[}float{]}
Time step between samples in seconds.

\item[{n\_stacks}] \leavevmode{[}int, optional{]}
Number of stacks used to produce the amplitude (the
default value is 1, denoting a single unstacked 
recording).

\item[{delay}] \leavevmode{[}float \{\textless{}=0.\}, optional{]}
Indicates the pre-event delay in seconds.

\end{description}

\item[{Returns:}] \leavevmode
Intialized TimeSeries object.

\item[{Raises:}] \leavevmode\begin{description}
\item[{ValueError:}] \leavevmode
If \sphinxtitleref{delay} is greater than 0.

\end{description}

\end{description}

\end{fulllineitems}

\index{bandpassfilter() (TimeSeries method)@\spxentry{bandpassfilter()}\spxextra{TimeSeries method}}

\begin{fulllineitems}
\phantomsection\label{\detokenize{index:sigpropy.TimeSeries.bandpassfilter}}\pysiglinewithargsret{\sphinxbfcode{\sphinxupquote{bandpassfilter}}}{\emph{flow}, \emph{fhigh}, \emph{order=5}}{}
Apply bandpass Butterworth filter to time series.
\begin{description}
\item[{Args:}] \leavevmode\begin{description}
\item[{flow}] \leavevmode{[}float{]}
Low-cut frequency (content below \sphinxtitleref{flow} is filtered).

\item[{fhigh}] \leavevmode{[}float{]}
High-cut frequency (content above \sphinxtitleref{fhigh} is filtered).

\item[{order}] \leavevmode{[}int, optional{]}
Filter order (default is 5th).

\end{description}

\item[{Returns:}] \leavevmode
\sphinxtitleref{None}, instead filters attribute \sphinxtitleref{amp}.

\end{description}

\end{fulllineitems}

\index{cosine\_taper() (TimeSeries method)@\spxentry{cosine\_taper()}\spxextra{TimeSeries method}}

\begin{fulllineitems}
\phantomsection\label{\detokenize{index:sigpropy.TimeSeries.cosine_taper}}\pysiglinewithargsret{\sphinxbfcode{\sphinxupquote{cosine\_taper}}}{\emph{width}}{}
Apply cosine taper to time series.
\begin{description}
\item[{Args:}] \leavevmode\begin{description}
\item[{width}] \leavevmode{[}float \{0.-1.\}{]}
Amount of the time series to be tapered.
0. is equal to a rectangular and 1. a Hann window.

\end{description}

\item[{Returns:}] \leavevmode
\sphinxtitleref{None}, applies cosine taper to attribute \sphinxtitleref{amp}.

\end{description}

\end{fulllineitems}

\index{detrend() (TimeSeries method)@\spxentry{detrend()}\spxextra{TimeSeries method}}

\begin{fulllineitems}
\phantomsection\label{\detokenize{index:sigpropy.TimeSeries.detrend}}\pysiglinewithargsret{\sphinxbfcode{\sphinxupquote{detrend}}}{}{}
Remove linear trend from time series.
\begin{description}
\item[{Returns:}] \leavevmode
\sphinxtitleref{None}, remove linear trend from attribute \sphinxtitleref{amp}.

\end{description}

\end{fulllineitems}

\index{from\_trace() (TimeSeries class method)@\spxentry{from\_trace()}\spxextra{TimeSeries class method}}

\begin{fulllineitems}
\phantomsection\label{\detokenize{index:sigpropy.TimeSeries.from_trace}}\pysiglinewithargsret{\sphinxbfcode{\sphinxupquote{classmethod }}\sphinxbfcode{\sphinxupquote{from\_trace}}}{\emph{trace}, \emph{n\_stacks=1}, \emph{delay=0}}{}
Initialize a TimeSeries object from a trace object.

This method is a more general method than \sphinxtitleref{from\_trace\_seg2}, 
as it does not attempt to extract any metadata from the Trace 
object.
\begin{description}
\item[{Args:}] \leavevmode\begin{description}
\item[{trace}] \leavevmode{[}Trace{]}
Refer to obspy documentation for more information
(\sphinxurl{https://github.com/obspy/obspy/wiki}).

\item[{n\_stacks}] \leavevmode{[}int, optional{]}
Number of stacks the time series represents, (default is
1, signifying a single unstacked time record).

\item[{delay}] \leavevmode{[}float \{\textless{}=0.\}, optional{]}
Denotes the pre-event delay, (default is zero, 
signifying no pre-event recording is included).

\end{description}

\item[{Returns:}] \leavevmode
Initialized TimeSeries object.

\end{description}

\end{fulllineitems}

\index{split() (TimeSeries method)@\spxentry{split()}\spxextra{TimeSeries method}}

\begin{fulllineitems}
\phantomsection\label{\detokenize{index:sigpropy.TimeSeries.split}}\pysiglinewithargsret{\sphinxbfcode{\sphinxupquote{split}}}{\emph{windowlength}}{}
Split time series into windows of duration \sphinxtitleref{windowlength}.
\begin{description}
\item[{Args:}] \leavevmode\begin{description}
\item[{windowlength}] \leavevmode{[}float{]}
Duration of desired window length in seconds. If 
\sphinxtitleref{windowlength} is not an integer multiple of \sphinxtitleref{dt}, the 
window length is rounded to up to the next integer
multiple of \sphinxtitleref{dt}.

\end{description}

\item[{Returns:}] \leavevmode
\sphinxtitleref{None}, reshapes attribute \sphinxtitleref{amp} into a 2D array 
where each row is a different consecutive time window and 
each column denotes a time step.

\item[{Note:}] \leavevmode
The last sample of each window is repeated as the first
sample of the following time window to ensure a logical
number of windows. Without this, a 10 minute record could
not be broken into 10 1-minute records.

\item[{Example:}] \leavevmode
\begin{sphinxVerbatim}[commandchars=\\\{\}]
\PYG{g+gp}{\PYGZgt{}\PYGZgt{}\PYGZgt{} }\PYG{k+kn}{import} \PYG{n+nn}{sigpropy} \PYG{k}{as} \PYG{n+nn}{sp}
\PYG{g+gp}{\PYGZgt{}\PYGZgt{}\PYGZgt{} }\PYG{k+kn}{import} \PYG{n+nn}{numpy} \PYG{k}{as} \PYG{n+nn}{np}
\PYG{g+gp}{\PYGZgt{}\PYGZgt{}\PYGZgt{} }\PYG{n}{amp} \PYG{o}{=} \PYG{n}{np}\PYG{o}{.}\PYG{n}{array}\PYG{p}{(}\PYG{p}{[}\PYG{l+m+mi}{0}\PYG{p}{,}\PYG{l+m+mi}{1}\PYG{p}{,}\PYG{l+m+mi}{2}\PYG{p}{,}\PYG{l+m+mi}{3}\PYG{p}{,}\PYG{l+m+mi}{4}\PYG{p}{,}\PYG{l+m+mi}{5}\PYG{p}{,}\PYG{l+m+mi}{6}\PYG{p}{,}\PYG{l+m+mi}{7}\PYG{p}{,}\PYG{l+m+mi}{8}\PYG{p}{,}\PYG{l+m+mi}{9}\PYG{p}{]}\PYG{p}{)}
\PYG{g+gp}{\PYGZgt{}\PYGZgt{}\PYGZgt{} }\PYG{n}{tseries} \PYG{o}{=} \PYG{n}{sp}\PYG{o}{.}\PYG{n}{TimeSeries}\PYG{p}{(}\PYG{n}{amp}\PYG{p}{,} \PYG{n}{dt}\PYG{o}{=}\PYG{l+m+mi}{1}\PYG{p}{)} 
\PYG{g+gp}{\PYGZgt{}\PYGZgt{}\PYGZgt{} }\PYG{n}{tseries}\PYG{o}{.}\PYG{n}{split}\PYG{p}{(}\PYG{l+m+mi}{2}\PYG{p}{)}
\PYG{g+gp}{\PYGZgt{}\PYGZgt{}\PYGZgt{} }\PYG{n}{tseries}\PYG{o}{.}\PYG{n}{amp}
\PYG{g+go}{array([[0, 1, 2],}
\PYG{g+go}{    [2, 3, 4],}
\PYG{g+go}{    [4, 5, 6],}
\PYG{g+go}{    [6, 7, 8]])}
\end{sphinxVerbatim}

\end{description}

\end{fulllineitems}

\index{time() (TimeSeries property)@\spxentry{time()}\spxextra{TimeSeries property}}

\begin{fulllineitems}
\phantomsection\label{\detokenize{index:sigpropy.TimeSeries.time}}\pysigline{\sphinxbfcode{\sphinxupquote{property }}\sphinxbfcode{\sphinxupquote{time}}}
Return time vector for TimeSeries object.

\end{fulllineitems}

\index{trim() (TimeSeries method)@\spxentry{trim()}\spxextra{TimeSeries method}}

\begin{fulllineitems}
\phantomsection\label{\detokenize{index:sigpropy.TimeSeries.trim}}\pysiglinewithargsret{\sphinxbfcode{\sphinxupquote{trim}}}{\emph{start\_time}, \emph{end\_time}}{}
Trim excess from time series in the half-open interval
{[}start\_time, end\_time).
\begin{description}
\item[{Args:}] \leavevmode\begin{description}
\item[{start\_time}] \leavevmode{[}float{]}
New time zero in seconds.

\item[{end\_time}] \leavevmode{[}float{]}
New end time in seconds. Note that the interval is
half-open.

\end{description}

\item[{Returns:}] \leavevmode
\sphinxtitleref{None},updates the attributes: \sphinxtitleref{n\_samples}, \sphinxtitleref{delay}, and
\sphinxtitleref{df}.

\item[{Raises:}] \leavevmode\begin{description}
\item[{IndexError:}] \leavevmode
If the \sphinxtitleref{start\_time} and \sphinxtitleref{end\_time} is illogical.
For example, \sphinxtitleref{start\_time} is before the start of the
\sphinxtitleref{delay} or after \sphinxtitleref{end\_time}, or the \sphinxtitleref{end\_time} is
after the end of the record.

\end{description}

\end{description}

\end{fulllineitems}

\index{zero\_pad() (TimeSeries method)@\spxentry{zero\_pad()}\spxextra{TimeSeries method}}

\begin{fulllineitems}
\phantomsection\label{\detokenize{index:sigpropy.TimeSeries.zero_pad}}\pysiglinewithargsret{\sphinxbfcode{\sphinxupquote{zero\_pad}}}{\emph{df}}{}
Append zeros to the end of the TimeSeries object to achieve a
desired frequency step.
\begin{description}
\item[{Args:}] \leavevmode\begin{description}
\item[{df}] \leavevmode{[}float{]}
Desired frequency step in Hz. Must be positive.

\end{description}

\item[{Returns:}] \leavevmode
\sphinxtitleref{None}, modifies attributes: \sphinxtitleref{amp}, \sphinxtitleref{n\_samples}, and
\sphinxtitleref{multiple}.

\item[{Raises:}] \leavevmode\begin{description}
\item[{TypeError:}] \leavevmode
If \sphinxtitleref{df} is not a float.

\item[{ValueError:}] \leavevmode
If \sphinxtitleref{df} is not positive.

\end{description}

\end{description}

\end{fulllineitems}


\end{fulllineitems}



\chapter{FourierTransform Class}
\label{\detokenize{index:fouriertransform-class}}\index{FourierTransform (class in sigpropy)@\spxentry{FourierTransform}\spxextra{class in sigpropy}}

\begin{fulllineitems}
\phantomsection\label{\detokenize{index:sigpropy.FourierTransform}}\pysiglinewithargsret{\sphinxbfcode{\sphinxupquote{class }}\sphinxbfcode{\sphinxupquote{FourierTransform}}}{\emph{amplitude}, \emph{frq}, \emph{fnyq=None}}{}
A class for manipulating Fourier transforms.
\begin{description}
\item[{Attributes:}] \leavevmode\begin{description}
\item[{frq}] \leavevmode{[}ndarray{]}
Frequency vector of the transform in Hz.

\item[{amp}] \leavevmode{[}ndarray{]}
The transform’s amplitude is in the same units as the input.
May be 1D or 2D. If 2D each row corresponds to a unique FFT,
where each column correpsonds to an entry in \sphinxtitleref{frq}.

\item[{fnyq}] \leavevmode{[}float{]}
The Nyquist frequency associated with the time series used
to generate the Fourier transform. Note this may or may not
be equal to \sphinxtitleref{frq{[}-1{]}}.

\end{description}

\end{description}
\index{\_\_init\_\_() (FourierTransform method)@\spxentry{\_\_init\_\_()}\spxextra{FourierTransform method}}

\begin{fulllineitems}
\phantomsection\label{\detokenize{index:sigpropy.FourierTransform.__init__}}\pysiglinewithargsret{\sphinxbfcode{\sphinxupquote{\_\_init\_\_}}}{\emph{amplitude}, \emph{frq}, \emph{fnyq=None}}{}
Initialize a FourierTransform object.
\begin{description}
\item[{Args:}] \leavevmode\begin{description}
\item[{amplitude}] \leavevmode{[}ndarray{]}
Fourier transform amplitude. Refer to attribute \sphinxtitleref{amp}
for more details.

\item[{frq}] \leavevmode{[}ndarray {]}
Linearly spaced frequency vector for Fourier transform.

\item[{fnyq}] \leavevmode{[}float, optional{]}
Nyquist frequency of Fourier Transform (by default the
maximum value of \sphinxtitleref{frq} vector is used).

\end{description}

\item[{Returns:}] \leavevmode
An initialized FourierTransform object.

\end{description}

\end{fulllineitems}

\index{fft() (FourierTransform static method)@\spxentry{fft()}\spxextra{FourierTransform static method}}

\begin{fulllineitems}
\phantomsection\label{\detokenize{index:sigpropy.FourierTransform.fft}}\pysiglinewithargsret{\sphinxbfcode{\sphinxupquote{static }}\sphinxbfcode{\sphinxupquote{fft}}}{\emph{amplitude}, \emph{dt}}{}
Compute the fast-Fourier transform (FFT) of a time series.
\begin{description}
\item[{Args:}] \leavevmode\begin{description}
\item[{amplitude}] \leavevmode{[}ndarray{]}
Denotes the time series amplitude. If \sphinxtitleref{amplitude} is 1D
each sample corresponds to a single time step. If
\sphinxtitleref{amplitude} is 2D each row corresponds to a particular
section of the time record (i.e., time window) and each
column corresponds to a single time step.

\item[{dt}] \leavevmode{[}float{]}
Denotes the time step between samples in seconds.

\end{description}

\item[{Returns:}] \leavevmode\begin{description}
\item[{Tuple of the form (frq, fft) where:}] \leavevmode\begin{description}
\item[{\sphinxtitleref{frq}}] \leavevmode{[}ndarray{]}
Positve frequency vector between zero and the
Nyquist frequency (if even) or near the Nyquist
(if odd) in Hz.

\item[{\sphinxtitleref{fft}}] \leavevmode{[}ndarray{]}
Complex amplitudes for the frequencies between zero
and the Nyquist (if even) or near the Nyquist 
(if odd) with units of the input ampltiude.
If \sphinxtitleref{amplitude} is a 2D array \sphinxtitleref{fft} will also be a 2D
array where each row is the FFT of each row of 
\sphinxtitleref{amplitude}.

\end{description}

\end{description}

\end{description}

\end{fulllineitems}

\index{from\_timeseries() (FourierTransform class method)@\spxentry{from\_timeseries()}\spxextra{FourierTransform class method}}

\begin{fulllineitems}
\phantomsection\label{\detokenize{index:sigpropy.FourierTransform.from_timeseries}}\pysiglinewithargsret{\sphinxbfcode{\sphinxupquote{classmethod }}\sphinxbfcode{\sphinxupquote{from\_timeseries}}}{\emph{timeseries}}{}
Create a FourierTransform object from a TimeSeries object.
\begin{description}
\item[{Args:}] \leavevmode\begin{description}
\item[{timeseries}] \leavevmode{[}TimeSeries {]}
TimeSeries object to be transformed.

\end{description}

\item[{Returns:}] \leavevmode
An initialized FourierTransform object.

\end{description}

\end{fulllineitems}

\index{imag() (FourierTransform property)@\spxentry{imag()}\spxextra{FourierTransform property}}

\begin{fulllineitems}
\phantomsection\label{\detokenize{index:sigpropy.FourierTransform.imag}}\pysigline{\sphinxbfcode{\sphinxupquote{property }}\sphinxbfcode{\sphinxupquote{imag}}}
Imaginary component of complex FFT amplitude.

\end{fulllineitems}

\index{mag() (FourierTransform property)@\spxentry{mag()}\spxextra{FourierTransform property}}

\begin{fulllineitems}
\phantomsection\label{\detokenize{index:sigpropy.FourierTransform.mag}}\pysigline{\sphinxbfcode{\sphinxupquote{property }}\sphinxbfcode{\sphinxupquote{mag}}}
Magnitude of complex FFT amplitude.

\end{fulllineitems}

\index{phase() (FourierTransform property)@\spxentry{phase()}\spxextra{FourierTransform property}}

\begin{fulllineitems}
\phantomsection\label{\detokenize{index:sigpropy.FourierTransform.phase}}\pysigline{\sphinxbfcode{\sphinxupquote{property }}\sphinxbfcode{\sphinxupquote{phase}}}
Phase of complex FFT amplitude in radians.

\end{fulllineitems}

\index{real() (FourierTransform property)@\spxentry{real()}\spxextra{FourierTransform property}}

\begin{fulllineitems}
\phantomsection\label{\detokenize{index:sigpropy.FourierTransform.real}}\pysigline{\sphinxbfcode{\sphinxupquote{property }}\sphinxbfcode{\sphinxupquote{real}}}
Real component of complex FFT amplitude.

\end{fulllineitems}

\index{resample() (FourierTransform method)@\spxentry{resample()}\spxextra{FourierTransform method}}

\begin{fulllineitems}
\phantomsection\label{\detokenize{index:sigpropy.FourierTransform.resample}}\pysiglinewithargsret{\sphinxbfcode{\sphinxupquote{resample}}}{\emph{minf}, \emph{maxf}, \emph{nf}, \emph{res\_type='log'}, \emph{inplace=False}}{}
Resample FourierTransform over a specified range.
\begin{description}
\item[{Args:}] \leavevmode\begin{description}
\item[{minf}] \leavevmode{[}float {]}
Minimum value of resample.

\item[{maxf}] \leavevmode{[}float{]}
Maximum value of resample.

\item[{nf}] \leavevmode{[}int{]}
Number of resamples.

\item[{res\_type}] \leavevmode{[}\{“log”, “linear”\}, optional{]}
Type of resampling, default value is \sphinxtitleref{log}.

\item[{inplace}] \leavevmode{[}bool, optional{]}
Determines whether resampling is done in place or 
if a copy is returned be returned. By default the
resampling is not done inplace (i.e., \sphinxtitleref{inplace=False}).

\end{description}

\item[{Returns:}] \leavevmode\begin{description}
\item[{If \sphinxtitleref{inplace=True}}] \leavevmode
\sphinxtitleref{None}, method edits the internal attribute \sphinxtitleref{amp}.

\item[{If \sphinxtitleref{inplace=False}}] \leavevmode
A tuple of the form (\sphinxtitleref{frequency}, \sphinxtitleref{amplitude})
where \sphinxtitleref{frequency} is the resampled frequency vector and 
\sphinxtitleref{amplitude} is the resampled amplitude vector if 
\sphinxtitleref{amp} is 1D or array if \sphinxtitleref{amp} is 2D.

\end{description}

\item[{Raises:}] \leavevmode\begin{description}
\item[{ValueError:}] \leavevmode
If \sphinxtitleref{maxf}, \sphinxtitleref{minf}, or \sphinxtitleref{nf} are illogical.

\item[{NotImplementedError:}] \leavevmode
If \sphinxtitleref{res\_type} is not amoung those options specified.

\end{description}

\end{description}

\end{fulllineitems}

\index{smooth\_konno\_ohmachi() (FourierTransform method)@\spxentry{smooth\_konno\_ohmachi()}\spxextra{FourierTransform method}}

\begin{fulllineitems}
\phantomsection\label{\detokenize{index:sigpropy.FourierTransform.smooth_konno_ohmachi}}\pysiglinewithargsret{\sphinxbfcode{\sphinxupquote{smooth\_konno\_ohmachi}}}{\emph{bandwidth=40.0}}{}
Apply Konno and Ohmachi smoothing.
\begin{description}
\item[{Args:}] \leavevmode\begin{description}
\item[{bandwidth}] \leavevmode{[}float, optional{]}
Width of smoothing window, by default this is set to 40.

\end{description}

\item[{Returns:}] \leavevmode
\sphinxtitleref{None}, modifies the internal attribute \sphinxtitleref{amp} to equal the
smoothed value of \sphinxtitleref{mag}.

\end{description}

\end{fulllineitems}


\end{fulllineitems}




\renewcommand{\indexname}{Index}
\printindex
\end{document}